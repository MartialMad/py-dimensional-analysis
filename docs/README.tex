\documentclass[11pt]{article}
\usepackage{amsmath,amssymb}
\usepackage{lmodern}
\usepackage{cite}
\usepackage{listings}

\title{py-dimensional-analysis}
\date{}

\begin{document}
\maketitle

\section{py-dimensional-analysis}
This Python package addresses physical dimensional analysis. In particular, \texttt{py-dimensional-analysis} calculates from a given system of (dimensional) variables those products that yield a desired target dimension.

% \begin{equation}
%     y = A^aB^bC_c
% \end{equation}

The following example illustrates how the variables mass, force, time and pressure must relate to each other in order to produce the dimension length*time.

\begin{lstlisting}[language=Python]
import danalysis as da

si = da.standard_systems.SI         # predefined standard units
s = da.Solver(
    {
        'a' : si.M,                 # [a] is mass
        'b' : si.L*si.M*si.T**-2,   # [b] is force (alt. si.F)
        'c' : si.T,                 # [c] is time
        'd' : si.Pressure           # [d] is pressure
    },
    si.L*si.T                       # target dimension
)
print(s.solve())
\end{lstlisting}
Which prints
\begin{lstlisting}
Found 2 variable products of variables
{
        a:Q(M),
        b:Q(L*M*T**-2),
        c:Q(T),
        d:Q(L**-1*M*T**-2)
}, each of dimension L*T:
        1: [a*c**-1*d**-1] = L*T
        2: [b**0.5*c*d**-0.5] = L*T
\end{lstlisting}

This library is based on \cite{szirtes2007applied}, and also incorporates ideas and examples from \cite{santiago2019first, sonin2001dimensional}.

\subsection{Solver}
The solver is based on the Buckingham's $\pi$ theorem. For more information, see \cite{szirtes2007applied}.

\subsection{References}
\bibliographystyle{alpha}
\begingroup
\renewcommand{\section}[2]{}%
\bibliography{biblio.bib}
\endgroup

\end{document}

%pandoc --citeproc -s README.tex -o README.md --to markdown_strict
