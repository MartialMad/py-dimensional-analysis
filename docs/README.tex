\documentclass[11pt]{article}
\usepackage{amsmath,amssymb}
\usepackage{lmodern}
\usepackage{cite}
\usepackage{listings}

\title{py-dimensional-analysis}
\date{}

\begin{document}
\maketitle

\section{py-dimensional-analysis}
This Python package addresses physical dimensional analysis. In particular, \texttt{py-dimensional-analysis} calculates from a given system of (dimensional) variables those products that yield a desired target dimension.

% \begin{equation}
%     y = A^aB^bC_c
% \end{equation}

The following example shows, a (single) relation between mass, force, and time produces length.

\begin{lstlisting}[language=Python]
import danalysis as da
import danalysis.standard_units as si

r = da.solve(
    {'a':si.M, 'b':si.F, 'c':si.T, 'd':si.pressure}, 
    si.L*si.T
)
print(r)
# Found 2 variable products generating dimension L*T:
#    1: [a*c**-1*d**-1] = L*T
#    2: [b**0.5*c*d**-0.5] = L*T
\end{lstlisting}

This library is based on \cite{szirtes2007applied} but also incorporates ideas from \cite{santiago2019first, sonin2001dimensional}.

\subsection{References}
\bibliographystyle{alpha}
\begingroup
\renewcommand{\section}[2]{}%
%\renewcommand{\chapter}[2]{}% for other classes
\bibliography{biblio.bib}
\endgroup

\end{document}

%pandoc --citeproc -s README.tex -o README.md --to markdown_strict